% preamble
%%%%%%%%%%%%%%%%%%%%%%%%%%%%%%%%%%%%%%%%%%%%%%%%%%%%%%%%%%%%%%%%%%%%%%%%%%%%%%%%%%%%
%  Allgemein
%\usepackage[latin1]{inputenc}
\usepackage[ngerman]{babel}     % Deutsche Sprache in automatmisch generiertem
\usepackage[utf8]{inputenc}     %  =E4 =F6 =FC =DF; danach  geht auch das ß richtig
\usepackage{longtable}
\usepackage{latexsym}           % Fuer recht seltene Zeichen
\usepackage{caption}            % Figure-Captions formatieren
\usepackage{sectsty}            % Section headings formatieren   
\usepackage{xparse}             % Für selbstgeschrieben Befehle 
\usepackage{lmodern}
 
%%%%%%%%%%%%%%%%%%%%%%%%%%%%%%%%%%%%%%%%%%%%%%%%%%%%%%%%%%%%%%%%%%%%%%%%%%%%%%%%%%%%
%%%%%%  Layout
	\usepackage[left=3cm,right=2.5cm,top=2.2cm,bottom=2.2cm,includeheadfoot]{geometry}

	%%%%%% Fonts
	\usepackage[T1]{fontenc} % T1 Schrift Encoding
	\usepackage{lmodern}
	%\usepackage[table]{xcolor}
	\usepackage[normalem]{ulem}	% unterstreichen im Text

	\usepackage[ngerman]{varioref} % Intelligente Querverweise

	%%%%%% Kopf und Fusszeile
	\usepackage[%
	   automark,         % automatische Aktualisierung der Kolumnentitel
	   markcase=ignoreuppercase,      % Grossbuchstaben verhindern
	]{scrlayer-scrpage}
   
	\pagestyle{scrheadings}
	% loescht voreingestellte Stile
	\clearmainofpairofpagestyles
	\clearplainofpairofpagestyles
	%\clearscrheadings
	%\clearscrplain
	% Was steht wo...
	   \ohead{\pagemark}
	   \ihead{\headmark}
	   \ofoot[\pagemark]{} % Außen unten: Seitenzahlen bei plain

	% Angezeigte Abschnitte im Header
	   \automark[section]{chapter} %[rechts]{links}
	%
	% Linien (moegliche Kombination mit Breiten)
	\setheadsepline{.4pt}[\color{black}]
      
	\setheadwidth[0pt]{text}
	\setfootwidth[0pt]{text}

%%%%%%%%%%%%%%%%%%%%%%%%%%%%%%%%%%%%%%%%%%%%%%%%%%%%%%%%%%%%%%%%%%%%%%%%%%%%%%%%%%%%
%  Bib und Links
	\usepackage[%
		square,	% for square brackets;
		comma,	% to use commas as separaters;
		numbers,	% for numerical citations;
		sort,		% orders multiple citations into the sequence in which they appear in the list of references;
		sort&compress,    % as sort but in addition multiple numerical citations
					   % are compressed if possible (as 3-6, 15);
	]{natbib}
	\bibliographystyle{dinat}

    \usepackage[margin=0pt]{subcaption} % specific form for captions
    \usepackage[autostyle]{csquotes} % helper for quoting


	\usepackage[
	   % Farben fuer die Links
	   colorlinks=true,         % Links erhalten Farben statt Kaeten
	   urlcolor=true,    % \href{...}{...} external (URL)
	   filecolor=true,  % \href{...} local file
	   linkcolor=true,  %\ref{...} and \pageref{...}
	   % Links
	   raiselinks=true,			 % calculate real height of the link
	   breaklinks,              % Links berstehen Zeilenumbruch
	   backref=page,            % Backlinks im Literaturverzeichnis (section, slide, page, none)
	   pagebackref=false,        % Backlinks im Literaturverzeichnis mit Seitenangabe
	   verbose,
	   hyperindex=true,         % backlinkex index
	   linktocpage=true,        % Inhaltsverzeichnis verlinkt Seiten
	   hyperfootnotes=false,     % Keine Links auf Fussnoten
	   % Bookmarks
	   bookmarks=true,          % Erzeugung von Bookmarks fuer PDF-Viewer
	   bookmarksopenlevel=1,    % Gliederungstiefe der Bookmarks
	   bookmarksopen=true,      % Expandierte Untermenues in Bookmarks
	   bookmarksnumbered=true,  % Nummerierung der Bookmarks
	   bookmarkstype=toc,       % Art der Verzeichnisses
	   % Anchors
	   plainpages=false,        % Anchors even on plain pages ?
	   pageanchor=true,         % Pages are linkable
	   % PDF Informationen
	   pdfpagelabels=true,      % set PDF page labels
	]{hyperref}
	
%%%%%%%%%%%%%%%%%%%%%%%%%%%%%%%%%%%%%%%%%%%%%%%%%%%%%%%%%%%%%%%%%%%%%%%%%%%%%%%%%%%%
%  Mathekram
	\usepackage{,amsfonts,amssymb,amsthm} % mathematical expression, symbols, fonts, and the whole theorem environment engine
	\usepackage[
	   intlimits,  % Like sumlimits, but for integral symbols.
		]{amsmath}	% math package with additional symbols
	\usepackage{icomma}			% Erlaubt die Benutzung von Kommas im Mathematikmodus
	\usepackage{units}				% for units
	\usepackage{caption}			
	\usepackage{pdfpages}			% include whole pdf files
	\usepackage{ragged2e}			% Besserer Flatternsatz (Linksbuendig, statt Blocksatz)
% Nützliche Abkürzungen
    \newcommand{\NN}{\mathbb{N}} % natürliche Zahlen
    \newcommand{\RR}{\mathbb{R}} % reelle Zahlen
    \newcommand{\QQ}{\mathbb{Q}} % rationale Zahlen
    \newcommand{\ZZ}{\mathbb{Z}} % ganze Zahlen
%%%%%%%%%%%%%%%%%%%%%%%%%%%%%%%%%%%%%%%%%%%%%%%%%%%%%%%%%%%%%%%%%%%%%%%%%%%%%%%%%%%%
%  Bilder
	\usepackage{tikz}
    %\usepackage{siunitx}
    \usepackage{standalone}
 	\usetikzlibrary{positioning, calc,quotes,angles,arrows,decorations.pathmorphing,decorations.pathreplacing,quotes} 
    \usetikzlibrary{shapes,backgrounds,fit,matrix}
    \usepackage{svg}                % für das einbinden von svg bildern
	\usepackage{float}              % Stellt die Option [H] fuer Floats zur Verfgung
   
	\usepackage[figure]{hypcap}     % Links auf Gleitumgebungen springen nicht zur Beschriftung,
	                                % sondern zum Anfang der Gleitumgebung
%    \usepackage{xcolor,graphicx}    % graphics and colorings
  
  \usepackage{graphicx}           %zusätzlich zu wrapfig zum einfügen von bildern der benötigt wird. 
    \usepackage{wrapfig}            %Textumflossene Bilder die nicht Textblock-Breite sind 
    

%%%%%%%%%%%%%%%%%%%%%%%%%%%%%%%%%%%%%%%%%%%%%%%%%%%%%%%%%%%%%%%%%%%%%%%%%%%%%%%%%%%%
%  Tabellen
	\usepackage{tabularx}		% Automatische Spaltenbreite
	\usepackage{booktabs} 		% bessere Abstaende innerhalb der Tabelle (Layout))
	\usepackage{multirow}       % Mehrfachspalten
    \usepackage{rotating}       % if you want to rotate any object
%%%%%%%%%%%%%%%%%%%%%%%%%%%%%%%%%%%%%%%%%%%%%%%%%%%%%%%%%%%%%%%%%%%%%%%%%%%%%%%%%%%%%
%  Pseudocode
\usepackage[ruled,vlined,linesnumbered,german,onelanguage]{algorithm2e}
\usepackage{algpseudocode}

%%%%%%%%%%%%%%%%%%%%%%%%%%%%%%%%%%%%%%%%%%%%%%%%%%%%%%%%%%%%%%%%%%%%%%%%%%%%%%%%%%%%%
% theorem und beweise

%\cref is a powerfull command that automatically states the current environment (e.g. Theorem, Lemma, etc.)
\usepackage[sort&compress,nameinlink,noabbrev,capitalize]{cleveref}
%\usepackage{scrextend} % super powerful tool that helps a bit with everything





%DIF PREAMBLE EXTENSION ADDED BY LATEXDIFF
%DIF UNDERLINE PREAMBLE %DIF PREAMBLE
\RequirePackage[normalem]{ulem} %DIF PREAMBLE
\RequirePackage{color}\definecolor{RED}{rgb}{1,0,0}\definecolor{BLUE}{rgb}{0,0,1} %DIF PREAMBLE
\providecommand{\DIFadd}[1]{{\protect\color{blue}\uwave{#1}}} %DIF PREAMBLE
\providecommand{\DIFdel}[1]{{\protect\color{red}\sout{#1}}}                      %DIF PREAMBLE
%DIF SAFE PREAMBLE %DIF PREAMBLE
\providecommand{\DIFaddbegin}{} %DIF PREAMBLE
\providecommand{\DIFaddend}{} %DIF PREAMBLE
\providecommand{\DIFdelbegin}{} %DIF PREAMBLE
\providecommand{\DIFdelend}{} %DIF PREAMBLE
\providecommand{\DIFmodbegin}{} %DIF PREAMBLE
\providecommand{\DIFmodend}{} %DIF PREAMBLE
%DIF FLOATSAFE PREAMBLE %DIF PREAMBLE
\providecommand{\DIFaddFL}[1]{\DIFadd{#1}} %DIF PREAMBLE
\providecommand{\DIFdelFL}[1]{\DIFdel{#1}} %DIF PREAMBLE
\providecommand{\DIFaddbeginFL}{} %DIF PREAMBLE
\providecommand{\DIFaddendFL}{} %DIF PREAMBLE
\providecommand{\DIFdelbeginFL}{} %DIF PREAMBLE
\providecommand{\DIFdelendFL}{} %DIF PREAMBLE
%DIF LISTINGS PREAMBLE %DIF PREAMBLE
\RequirePackage{listings} %DIF PREAMBLE
\RequirePackage{color} %DIF PREAMBLE
\lstdefinelanguage{DIFcode}{ %DIF PREAMBLE
%DIF DIFCODE_UNDERLINE %DIF PREAMBLE
  moredelim=[il][\color{red}\sout]{\%DIF\ <\ }, %DIF PREAMBLE
  moredelim=[il][\color{blue}\uwave]{\%DIF\ >\ } %DIF PREAMBLE
} %DIF PREAMBLE
\lstdefinestyle{DIFverbatimstyle}{ %DIF PREAMBLE
	language=DIFcode, %DIF PREAMBLE
	basicstyle=\ttfamily, %DIF PREAMBLE
	columns=fullflexible, %DIF PREAMBLE
	keepspaces=true %DIF PREAMBLE
} %DIF PREAMBLE
\lstnewenvironment{DIFverbatim}{\lstset{style=DIFverbatimstyle}}{} %DIF PREAMBLE
\lstnewenvironment{DIFverbatim*}{\lstset{style=DIFverbatimstyle,showspaces=true}}{} %DIF PREAMBLE
%DIF END PREAMBLE EXTENSION ADDED BY LATEXDIFF






