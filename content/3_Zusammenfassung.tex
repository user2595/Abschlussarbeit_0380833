\chapter*{Abstract}
This thesis aims to assess the impact of angle defects on the termination of the intrinsic Delaunay-Refinement (iDR). The emergence of an angular defect on a surface means that the angles around a point do not add up to $2\pi$. This leads to the following research question: What are the conditions under which the proof of termination of planar Delaunay-Refinement (pDR) algorithms may be applied to iDR for closed surface triangulations without boundary? To answer this question, the proof of termination for known DR algorithms is analyzed in the Euclidean plane. This is done to identify possible problems that could emerge in the process. Subsequently, ways to overcome these problems are introduced. Our evaluation shows that the only angular defects that become a problem with regard to the proof of termination are those with a magnitude of $\frac{5}{3}\pi$ or greater. It is demonstrated that defects of smaller magnitude generally have a valid proof of termination. This shows that the Delaunay-Refinement’s range of applications can be extended to surface triangulations if the angular defect is less than $\frac{5}{3}\pi$. Finally, it is shown that the presumption that rules regarding handling of segments, grade and size optimality as well as relevant Lemmas in the Euclidean plane can also be used for surfaces is a reasonable notion.

\chapter*{Zusammenfassung}
Ziel der vorliegenden Arbeit ist die Untersuchung des Einflusses eines Winkeldefekts auf die Terminierung des intrinsischen Delaunay-Refinement (iDR).  Das Auftreten eines Winkeldefekts bei Oberflächen bedeutet, dass sich die Winkel um einen Punkt zu weniger als  $2\pi$ addieren. \\
    Daraus ergibt sich folgende Forschungsfrage: Unter welchen Bedingungen lässt sich der Terminierungsbeweis vom planaren Delaunay-Refinement (pDR)  auf das iDR für geschlossene Oberflächentriangulierungen  übertragen.\\
    Um diese Frage zu beantworten, wird zuerst der Terminierungsbeweis des Delaunay-Refinement für die euklidische Ebene analysiert. Dabei werden die Probleme ermittelt, die sich bei der Übertragung des Beweises auf iDR ergeben könnten.
 Im Anschluss wird gezeigt, wie die ermittelte Probleme behandelt werden können.\\
    Die Auswertungen ergaben, dass der Winkeldefekt erst ein Problem für die Terminierung darstellt, wenn dieser größer als $\frac{5}{3}\pi$  ist. Für einen kleineren Defekt lässt sich die Terminiertheit beweisen.\\
    Das zeigt, dass für einen Winkeldefekt kleiner als $\frac{5}{3} \pi$ das Delaunay-Refinement auf Oberflächentriangulierungen erweitert werden kann.\\
    Desweiteren wird in dieser Arbeit die Vermutung erläutert, weshalb für den Umgangen mit Segmenten und Abstufungs- sowie Größenoptimalität die selben Lemmas wie für die euklidische Ebene gelten könnten.